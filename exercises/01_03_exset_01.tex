{In Exercises}
{, find the solution of the given problem by:
\begin{description}
\item		[(a)]creating an appropriate system of linear equations
\item		[(b)]forming the augmented matrix that corresponds to this system
\item		[(c)]putting the augmented matrix into \rref
\item		[(d)]interpreting the \rref\ of the matrix as a solution
\end{description}}
\exinput{exercises/01_01_ex_19}
\exinput{exercises/01_01_ex_21}
\exinput{exercises/01_01_ex_17}
\exinput{exercises/01_01_ex_18}
\exinput{exercises/01_01_ex_20}