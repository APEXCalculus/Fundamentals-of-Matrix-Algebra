{$\tta = \left[\hskip -3pt \begin{array}{cc} 1&-2\\  0&1\end {array} \hskip -3pt
 \right]$, $\vb=\left[\hskip -3pt \begin{array}{c} 0\\  -5\end{array}\hskip -3pt \right]$}
{\begin{enumerate}
\item	 $\vx=\bmx{c}0\\0\emx$
\item	 $\vx=\left[\hskip -3pt \begin{array}{c} -10\\  -5\end {array} \hskip -3pt
\right]$
\end{enumerate}
 }







