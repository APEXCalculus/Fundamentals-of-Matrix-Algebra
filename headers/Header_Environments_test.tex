


\newlength{\topmarginlength}
\newlength{\bottommarginlength}
\newlength{\oddpagemarginlength}
\newlength{\evenpagemarginlength}
\newlength{\marginlinelength}

\newcounter{examplestartpageref}
\newcounter{pagedifference}
\newcounter{other}

\setlength{\marginlinelength}{5pt}
\setlength{\topmarginlength}{-1in-\voffset}
\setlength{\bottommarginlength}{-1in-\textheight-2\baselineskip}
\setlength{\oddpagemarginlength}{1in+\hoffset+\oddsidemargin-2\marginlinelength}
\setlength{\evenpagemarginlength}{1in+\hoffset+\evensidemargin-2\marginlinelength}

\newcommand{\colorlinecolor}{blue!95!black!30}
\newcommand{\bwlinecolor}{black!30}
\ifthenelse{\boolean{in_color}}{\newcommand{\thelinecolor}{\colorlinecolor}}{\newcommand{\thelinecolor}{\bwlinecolor}}
\newcommand{\linestyle}{thick}

\newboolean{showexamplelines}


\newcounter{examplecounter}
\setcounter{examplecounter}{0}

\newcounter{definitioncounter}
\setcounter{definitioncounter}{0}

\newcounter{theoremcounter}
\setcounter{theoremcounter}{0}

\newcounter{keyideacounter}
\setcounter{keyideacounter}{0}

\newlength{\specialboxlength}
\newlength{\specialboxtitlelength}
\newlength{\specialboxinnerseplength}

\setlength{\specialboxtitlelength}{75pt}
\setlength{\specialboxinnerseplength}{15pt}
\setlength{\specialboxlength}{\textwidth-\specialboxtitlelength-2\specialboxinnerseplength}

\makeatletter
\newcommand{\example}{\@ifstar \examplestarred \examplenostar}
\makeatother

%% This is the no-star (regular) version of
%% the example command.
\newcommand{\examplenostar}[3]{%
\noindent%
\hskip -2\marginlinelength%
\parbox{\marginlinelength}{%
\begin{tikzpicture}[remember picture,overlay]%
\draw (0,0) circle (1pt) node (#1) {};%
\end{tikzpicture}%
}% ends parbox where line begins
\hskip \marginlinelength% puts us back in line
\parbox{80pt}{{\bf Example \refstepcounter{examplecounter}\theexamplecounter}}% ends parbox
\label{#1}%
\ifthenelse{\pageref{#1}=\pageref{e#1}}% if the beginning and end are on the same page
{%
#2 \vskip \baselineskip%
\parbox{65pt}{\textsc{\small\bfseries Solution}}% 
#3  \label{e#1}% writes the full example then draws the lines
\ifthenelse{\boolean{showexamplelines}}{%
\begin{tikzpicture}[remember picture,overlay] \draw (0,0) circle (1pt) node (e#1) {};
\draw [\linestyle,\thelinecolor] (#1) -- (#1.south |- e#1.south) -- ++(10pt,0);
\end{tikzpicture}
}{}% ends if/then/else show lines
}% ends if beginning and end are on same page.
% next is if they are on different pages
{% first draw line from start to bottom of page
\ifthenelse{\boolean{showexamplelines}}{%
\begin{tikzpicture}[remember picture,overlay]% 
\ifthenelse{\isodd{\pageref{#1}}}% draws lines based on whether on an even or odd page %
{\node [xshift=\oddpagemarginlength,yshift=\bottommarginlength](bottomleft) at (current page.north west)  {};}
{\node [xshift=\evenpagemarginlength,yshift=\bottommarginlength](bottomleft) at (current page.north west)  {};}
\draw [\thelinecolor,\linestyle] (#1) -- (bottomleft);
\end{tikzpicture}
}{}% ends if/then/else show lines
% end drawing of line
#2 \vskip \baselineskip%
\parbox{65pt}{\textsc{\small \bfseries Solution}}%
#3  \label{e#1}% now write out full example
% now draw line from end to top of page
\ifthenelse{\boolean{showexamplelines}}{%
\begin{tikzpicture}[remember picture,overlay] \draw (0,0) circle (1pt) node (e#1) {};
\ifthenelse{\isodd{\pageref{e#1}}}% draws lines based on whether on an even or odd page %
{\node [xshift=\oddpagemarginlength,yshift=\topmarginlength](topleft) at (current page.north west)  {};}
{\node [xshift=\evenpagemarginlength,yshift=\topmarginlength](topleft) at (current page.north west)  {};}
\draw [\thelinecolor,\linestyle] (topleft)--(e#1.south -| topleft) -- ++(10pt,0);
\end{tikzpicture}%
}{}% ends if/then/else show lines
}% ends the check for same page or not
%
}%ends the definition of example

% This is the starred version of example.
% It does not include a ``solution'' and only
% takes 2 arguments.

\newcommand{\examplestarred}[2]{%
\noindent%
\hskip -2\marginlinelength%
\parbox{\marginlinelength}{%
\begin{tikzpicture}[remember picture,overlay]%
\draw (0,0) circle (1pt) node (#1) {};%
\end{tikzpicture}%
}% ends parbox where line begins
\hskip \marginlinelength% puts us back in line
\parbox{80pt}{{\bf Example \refstepcounter{examplecounter}\theexamplecounter}}% ends parbox
\label{#1}%
\ifthenelse{\pageref{#1}=\pageref{e#1}}% if the beginning and end are on the same page
{%
#2 \label{e#1}% writes the full example then draws the lines
\ifthenelse{\boolean{showexamplelines}}{%
\begin{tikzpicture}[remember picture,overlay] \draw (0,0) circle (1pt) node (e#1) {};
\draw [\linestyle,\thelinecolor] (#1) -- (#1.south |- e#1.south) -- ++(10pt,0);
\end{tikzpicture}
}{}% ends if/then/else show lines
}% ends if beginning and end are on same page.
% next is if they are on different pages
{% first draw line from start to bottom of page
\ifthenelse{\boolean{showexamplelines}}{%
\begin{tikzpicture}[remember picture,overlay]% 
\ifthenelse{\isodd{\pageref{#1}}}% draws lines based on whether on an even or odd page %
{\node [xshift=\oddpagemarginlength,yshift=\bottommarginlength](bottomleft) at (current page.north west)  {};}
{\node [xshift=\evenpagemarginlength,yshift=\bottommarginlength](bottomleft) at (current page.north west)  {};}
\draw [\thelinecolor,\linestyle] (#1) -- (bottomleft);
\end{tikzpicture}
}{}% ends if/then/else show lines
% end drawing of line
#2 \label{e#1}% now write out full example
% now draw line from end to top of page
\ifthenelse{\boolean{showexamplelines}}{%
\begin{tikzpicture}[remember picture,overlay] \draw (0,0) circle (1pt) node (e#1) {};
\ifthenelse{\isodd{\pageref{e#1}}}% draws lines based on whether on an even or odd page %
{\node [xshift=\oddpagemarginlength,yshift=\topmarginlength](topleft) at (current page.north west)  {};}
{\node [xshift=\evenpagemarginlength,yshift=\topmarginlength](topleft) at (current page.north west)  {};}
\draw [\thelinecolor,\linestyle] (topleft)--(e#1.south -| topleft) -- ++(10pt,0);
\end{tikzpicture}%
}{}% ends if/then/else show lines
}% ends the check for same page or not
%
}%ends the definition of example


% Draws a line on a page that doesn't contain either
% the beginning or end of an example.
% It has one arguments, the  label
% of the current example. 
\newcommand{\drawexampleline}[1]{
\setcounterpageref{examplestartpageref}{#1}%
\setcounter{pagedifference}{\thepage-\theexamplestartpageref}
\ifthenelse{\isodd{\theexamplestartpageref}}% example is on an odd page
       {\ifthenelse{\isodd{\thepagedifference}}% is the difference between this page
       % and the start even or odd?
       % odd: then plots on different side of page
       {\begin{tikzpicture}[remember picture,overlay]
        \node [xshift=\evenpagemarginlength,yshift=\topmarginlength](tleft) at (current page.north west)  {};
        \draw [\thelinecolor,thick] (tleft) -- ++(0,-\textheight);
        \end{tikzpicture}
        }% ends if difference is odd
        {% difference is even
        \begin{tikzpicture}[remember picture,overlay]
        \node [xshift=\oddpagemarginlength,yshift=\topmarginlength](tleft) at (current page.north west)  {};
        \draw [\thelinecolor,thick] (tleft) -- ++(0,-\textheight-2\baselineskip);
        \end{tikzpicture}
        }% ends difference is even and start page is odd
}% ends start page is odd
       {\ifthenelse{\isodd{\thepagedifference}}% is the difference between this page
       % and the start even or odd?
       % odd: then plots on different side of page
       {\begin{tikzpicture}[remember picture,overlay]
        \node [xshift=\oddpagemarginlength,yshift=\topmarginlength](tleft) at (current page.north west)  {};
        \draw [\thelinecolor,thick] (tleft) -- ++(0,-\textheight);
        \end{tikzpicture}
        }% ends if difference is odd
        {% difference is even
        \begin{tikzpicture}[remember picture,overlay]
        \node [xshift=\evenpagemarginlength,yshift=\topmarginlength](tleft) at (current page.north west)  {};
        \draw [\thelinecolor,thick] (tleft) -- ++(0,-\textheight);
        \end{tikzpicture}
        }% ends difference is even and start page is even
}% ends start page is even
} % End show lines



%
%Define style for Definitions, Theorems and Key Ideas  %%%%%%%%%%%%%%%%%%%%%%%%%%%%%%%%%%%%%%%%%%%%
%%%%%%%%%%%%%%%%%%%%%%%%%%%%%%%%%%%%%%%%%%%%%%%%%%%%%%%%%%%%%%%%%%%%%%%%%%%%%%%%%%%%%%%%%%%%%%%%%%%
%

%\newcounter{definitioncounter}
%\newcounter{theoremcounter}
%\newcounter{ideacounter}
%\newcounter{excounter}

\definecolor{myblue}{rgb}{.7,.7,1}
\definecolor{mygreen}{rgb}{.7,1,.7}
\definecolor{myred}{rgb}{1,.7,.7}
\definecolor{myyellow}{rgb}{.9,.9,0}

\newtheoremstyle{mystyle}% Generic Theorem Style
  {15pt}%      Space above
  {15pt}%      Space below
  {\sffamily}%         Body font
  {}%         Indent amount (empty = no indent, \parindent = para indent)
  {\sffamily\bfseries}% Thm head font
  {}%        Punctuation after thm head
  {1.5em}%     Space after thm head: " " = normal interword space;
        %       \newline = linebreak
  {}%         Thm head spec (can be left empty, meaning `normal')

\newtheoremstyle{myexamplestyle}% Used for the Example environments
  {15pt}%      Space above
  {5pt}%      Space below
  {}%         Body font
  {}%         Indent amount (empty = no indent, \parindent = para indent)
  {\bfseries}% Thm head font
  {}%        Punctuation after thm head
  {1.5em}%     Space after thm head: " " = normal interword space;
        %       \newline = linebreak
  {}%         Thm head spec (can be left empty, meaning `normal')

\theoremstyle{mystyle}

%\newtheorem{deff}{Definition}

%\newcommand{\definition}[2]{\begin{deff}\label{#1} \fcolorbox{blue}{myblue}{ \begin{minipage}[t]{265pt}#2\end{minipage}}\end{deff}}


\newcommand{\definition}[2]{%
\vskip\baselineskip
\noindent\refstepcounter{definitioncounter}\label{#1}%
\begin{tikzpicture}
\draw (0,0) node[anchor=north west,text width=80pt,inner sep=0pt]{\bf Definition \thedefinitioncounter};
\ifthenelse{\boolean{in_color}} %first, in color
{\draw (\specialboxtitlelength,0) node[rectangle,text width = \specialboxlength,very thick,inner sep=\specialboxinnerseplength,draw = yellow!95!black!60,top color = white!95!yellow,bottom color = yellow!90!black!30,text justified,anchor=north west,yshift=\specialboxinnerseplength] {#2};}%if not in color
{\draw (\specialboxtitlelength,0) node[rectangle,text width = \specialboxlength,very thick,inner sep=\specialboxinnerseplength,draw,text justified,anchor=north west,yshift=\specialboxinnerseplength] {#2};}
\end{tikzpicture}
}

%%%%%%%%%%%%%%%%%%%%%%%%%%%%%%%%%%%%%%%%%%%%%%%%%%%%%%%%%%%%%%%%%%%%%%%%%%%%%%%%%%%%%%%%
%%%%%%%%%%%%%%%%%%%%%%%%%%%%%%%%%%%%%%%%%%%%%%%%%%%%%%%%%%%%%%%%%%%%%%%%%%%%%%%%%%%%%%%%

\newcommand{\colortheorem}[2]{%
\vskip\baselineskip
\noindent\refstepcounter{theoremcounter}\label{#1}%
\begin{tikzpicture}
\draw (0,0) node[anchor=north west,text width=80pt,inner sep=0pt]{\bf Theorem \thetheoremcounter};
\ifthenelse{\boolean{in_color}} %first, in color
{\draw (\specialboxtitlelength,0) node[rectangle,text width = \specialboxlength,very thick,inner sep=\specialboxinnerseplength,draw = green!30!black!50,top color = white!95!green,bottom color = green!60!black!20,text justified,anchor=north west,yshift=\specialboxinnerseplength] {#2};}%if not in color
{\draw (\specialboxtitlelength,0) node[rectangle,text width = \specialboxlength,very thick,inner sep=\specialboxinnerseplength,draw,text justified,anchor=north west,yshift=\specialboxinnerseplength] {#2};}
\end{tikzpicture}
}


%%%%%%%%%%%%%%%%%%%%%%%%%%%%%%%%%%%%%%%%%%%%%%%%%%%%%%%%%%%%%%%%%%%%%%%%%%%%%%%%%%%%%%%%
%%%%%%%%%%%%%%%%%%%%%%%%%%%%%%%%%%%%%%%%%%%%%%%%%%%%%%%%%%%%%%%%%%%%%%%%%%%%%%%%%%%%%%%%

\newcommand{\idea}[2]{%
\vskip\baselineskip
\noindent\refstepcounter{keyideacounter}\label{#1}%
\begin{tikzpicture}
\draw (0,0) node[anchor=north west,text width=80pt,inner sep=0pt]{\bf Key Idea \thekeyideacounter};
\ifthenelse{\boolean{in_color}} %first, in color
{\draw (\specialboxtitlelength,0) node[rectangle,text width = \specialboxlength,very thick,inner sep=\specialboxinnerseplength,draw = red!30!black!50,top color = white!95!red,bottom color = red!60!black!20,text justified,anchor=north west,yshift=\specialboxinnerseplength] {#2};}%if not in color
{\draw (\specialboxtitlelength,0) node[rectangle,text width = \specialboxlength,very thick,inner sep=\specialboxinnerseplength,draw,text justified,anchor=north west,yshift=\specialboxinnerseplength] {#2};}
\end{tikzpicture}
}

%%%%%%%%%%%%%%%%%%%%%%%%%%%%%%%%%%%%%%%%%%%%%%%%%%%%%%%%%%%%%%%%%%%%%%%%%%%%%%%%%%%%%%%%
%%%%%%%%%%%%%%%%%%%%%%%%%%%%%%%%%%%%%%%%%%%%%%%%%%%%%%%%%%%%%%%%%%%%%%%%%%%%%%%%%%%%%%%%

\newcommand{\asyouread}[1]{\begin{tikzpicture}
\ifthenelse{\boolean{in_color}}{\node [preaction={fill=black,opacity=.5,transform canvas={xshift=1mm,yshift=-1mm}}, right color=blue!80!black!30, left color=blue!80] at (0,0) {\textcolor{white}{\textsf{\textit{AS YOU READ $\ldots$}}}};}
{\node [preaction={fill=black,opacity=.5,transform canvas={xshift=1mm,yshift=-1mm}}, right color=black!30, left color=black!10] at (0,0) {\textcolor{white}{\textsf{\textit{AS YOU READ $\ldots$}}}};}
\end{tikzpicture}
\begin{enumerate}
#1
\end{enumerate}
\vskip 20pt}
