%%%%
%% These are low level LaTeX commands that determine the 
%% look of the Chapter and Section headings.
%% Note the use in the chapter part of an external file
%% that contains graphics for each chapter start.
%%%%

%%%%
%% Commands for the header, utilizing the fancyhdr
%% (fancy header) package
%%%%

\pagestyle{fancy}
\fancyhead{}
\fancyfoot{}
\renewcommand{\chaptermark}[1]{\markboth{\chaptername\ \thechapter\ \ \ \ {#1}}{}}
\renewcommand{\sectionmark}[1]{\markright{\thesection\ \ \ \  #1}}
\fancyhf{}         %Clears all header and footer fields, in preparation.

\fancyhfoffset[LE,RO]{30pt}
\fancyfoot[LE,RO]{\textbf{\thepage}} %Displays the page number in bold in the header,
                       % to the left on even pages and to the right on odd pages.        
\fancyhead[LE]{\nouppercase{\leftmark}}
      %Displays the upper-level (chapter) information---
      % as determined above---in non-upper case in the header, 
      %to the right on even pages.
\fancyhead[RO]{\rightmark}
			%Displays the lower-level (section) information---as
      % determined above---in the header, to the left on odd pages.
\renewcommand{\headrulewidth}{0pt}
\renewcommand{\footrulewidth}{0pt}
	%Underlines the header and footer. (Set to 0pt if not required).



\makeatletter
\def\@makechapterhead#1{%
  {\parindent \z@ \raggedright \reset@font
    \large
    \resizebox{1.6cm}{2cm}{\thechapter}\ \ %
		\if \thechapter 1
%
\begin{tikzpicture}
%Chapter 1 Pic 1
\begin{scope}
\draw (-1,0)--(2,0);
\draw (0,-.5)--(0,3);
\draw [thick] (-.5,-.5)--(1.5,1.5);
\draw [thick] (-.5,2.5)--(1.5,.5);
\draw  (-1.1,-.6)--(2.1,-.6)--(2.1,3.1)--(-1.1,3.1)--cycle;
\end{scope}

%Chapter 1 pic 2
\begin{scope}[shift={(3.4,0)}]
\draw (-1,0)--(2,0);
\draw (0,-.5)--(0,3);
\draw [thick] (-.5,1.5)--(1.5,-.5);
\draw [thick] (-.5,2.5)--(1.5,.5);
\draw  (-1.1,-.6)--(2.1,-.6)--(2.1,3.1)--(-1.1,3.1)--cycle;
\end{scope}

%Chapter 1 pic 3
\begin{scope}[shift={(6.8,0)}]
\draw (-1,0)--(2,0);
\draw (0,-.5)--(0,3);
\draw [very thick] (-.5,1.5)--(1.5,-.5);
%\draw [thick] (-.5,2.5)--(1.5,.5);
\draw (-1.1,-.6)--(2.1,-.6)--(2.1,3.1)--(-1.1,3.1)--cycle;
\end{scope}
\end{tikzpicture}
%
\else \if \thechapter 2
%
\begin{tikzpicture}[>=latex]

%chapter 2 pic 1
\begin{scope}
\draw (-1,0)--(2,0);
\draw (0,-.5)--(0,3);
\draw [thick,->] (0,0)--(-.5,1.5);
\draw [thick,->] (0,0)--(1.5,.75);
\draw [thick,->,dashed,gray] (-.5,1.5)--(1,2.25);
\draw [thick,->,dashed,gray] (1.5,.75)--(1,2.25);
\draw [thick,->] (0,0)--(1,2.25);
\draw  (-1.1,-.6)--(2.1,-.6)--(2.1,3.1)--(-1.1,3.1)--cycle;
\end{scope}

%Chapter 2 pic 2
\begin{scope}[shift={(3.4,0)}]
\draw (-1,0) -- (2,0);
\draw (0,-.5)--(0,3);
\draw [thick] (0,1) circle (.5cm);
\draw [thick, gray, rotate around={45:(1.1,1)}] (1.1,1) ellipse (.6cm and .4cm);
%\draw [thick,gray] (.6,1) node [right] {$A'$} .. (.6,.5) .. (1.65,1) .. (1.65,1.5) node [below left] {$B'$} ..cycle;
%\node [above] at (.817cm,1.28cm) {$A$};
%\fill (.817cm,1.28cm) circle (.04cm); 
\fill [rotate around={45:(1.1,1)}](.5,1) circle (.04cm) node [above right] {$A'$};
\fill [rotate around={45:(1.1,1)}](1.1,1.4) circle (.04cm) node [above ] {$B'$};

\fill (0,1.5) circle (.04) node [above right] {$B$}; 
\fill (-.5,1) circle (.04) node [right] {$A$};

\draw  (-1.1,-.6)--(2.1,-.6)--(2.1,3.1)--(-1.1,3.1)--cycle;
\end{scope}

%chapter 2 pic 3
\begin{scope}[shift={(6.8,0)}]
\draw (-1,0)--(2,0);
\draw (0,-.5)--(0,3);

\draw (-1,-.5)--(1.8,.9);
\draw [->,thick] (0,0)--(1,.5);
\draw [->,thick,dashed] (0,0)--(-.5,1.5);
\draw (-1,1.25)--(1.8,2.65);
\draw [->,thick] (-.5,1.5)--(.5,2);

\draw  (-1.1,-.6)--(2.1,-.6)--(2.1,3.1)--(-1.1,3.1)--cycle;\end{scope}
\end{tikzpicture}
%
\else \if \thechapter 3
%
\begin{tikzpicture}[>=latex]

%chapter 3 pic 1
\begin{scope}
\draw (-1,0)--(2,0);
\draw (0,-.5)--(0,3);
\draw  (-1.1,-.6)--(2.1,-.6)--(2.1,3.1)--(-1.1,3.1)--cycle;

\draw [->,thick] (0,0)--(-.5,1.5);
\draw [->,thick] (0,0)--(1,.5);
\draw [->,thick,dashed,gray] (-.5,1.5)--(.5,2);
\draw [->,thick,dashed,gray] (1,.5)--(.5,2);

\end{scope}

%chapter 3 pic 2
\begin{scope}[shift={(3.4,0)}]
\draw (-1,0) -- (2,0);
\draw (0,-.5)--(0,3);
\draw  (-1.1,-.6)--(2.1,-.6)--(2.1,3.1)--(-1.1,3.1)--cycle;

\draw [->,thick] (0,0)--(-.5,1);
\draw [->,thick] (0,0)--(1.5,.5);
\draw [->,thick,dashed,gray] (-.5,1)--(1,1.5);
\draw [->,thick,dashed,gray] (1.5,.5)--(1,1.5);

\end{scope}

%chapter 3 pic 3
\begin{scope}[shift={(6.8,0)}]
\draw (-1,0)--(2,0);
\draw (0,-.5)--(0,3);
\draw  (-1.1,-.6)--(2.1,-.6)--(2.1,3.1)--(-1.1,3.1)--cycle;

\draw [->,thick] (0,0)--(1.25,-.25);
\draw [->,thick] (0,0)--(-.25,2.25);
\draw [->,thick,dashed,gray] (1.25,-.25)--(1,2);
\draw [->,thick,dashed,gray] (-.25,2.25)--(1,2);
\end{scope}

\end{tikzpicture}
%
\else \if \thechapter 4
%
\begin{tikzpicture}[>=latex]

%chapter 4 pic 1
\begin{scope}
\draw (-1,0)--(2,0);
\draw (0,-.5)--(0,3);
\draw  (-1.1,-.6)--(2.1,-.6)--(2.1,3.1)--(-1.1,3.1)--cycle;

\draw [->,thick] (0,0)--(1,.5);
\draw [->,thick] (0,0)--(.5,1);
\draw [->,thick,dashed,gray] (1,.5)--(1.5,1.5);
\draw [->,thick,dashed,gray] (.5,1)--(1.5,1.5);

\end{scope}

%chapter 4 pic 2
\begin{scope}[shift={(3.4,0)}]
\draw (-1,0) -- (2,0);
\draw (0,-.5)--(0,3);
\draw  (-1.1,-.6)--(2.1,-.6)--(2.1,3.1)--(-1.1,3.1)--cycle;

\draw [->,thick] (0,0)--(-1,1);
\draw [->,thick] (0,0)--(1,1);
\draw (0,.5) circle (.5);
\fill (0,0) circle (.04); 
\fill (-.5,.5) circle (.04); 
\fill (.5,.5) circle (.04); 
\fill (0,1) circle (.04); 

\end{scope}

%chapter 4 pic 3
\begin{scope}[shift={(6.8,0)}]
\draw (-1,0)--(2,0);
\draw (0,-.5)--(0,3);
\draw  (-1.1,-.6)--(2.1,-.6)--(2.1,3.1)--(-1.1,3.1)--cycle;

\draw [->,thick] (0,0)--(-.5,.5);
\draw [->,thick] (0,0)--(1.5,1.5);
\fill (0,0) circle (.04); 
\fill (-.25,.25) circle (.04); 
\fill (.75,.75) circle (.04); 
\fill (.5,1) circle (.04); 
%\draw (.25,.5) circle (.55);

\draw [rotate around={45:(.25,.5)}] (.25,.5) ellipse (.57 and .5);

\end{scope}

\end{tikzpicture}
%
\else \if \thechapter 5
%
\begin{tikzpicture}[>=latex]

%chapter 5 pic 1
\begin{scope}
\draw (-1,0)--(2,0);
\draw (0,-.5)--(0,3);

\draw (-1,-.5)--(1.8,.9);
\draw [->,thick] (0,0)--(1,.5);
\draw [->,thick,dashed] (0,0)--(-.5,1.5);
\draw (-1,1.25)--(1.8,2.65);
\draw [->,thick] (-.5,1.5)--(.5,2);

\draw  (-1.1,-.6)--(2.1,-.6)--(2.1,3.1)--(-1.1,3.1)--cycle;
\end{scope}

%chapter 5 pic 2
\begin{scope}[shift={(3.4,0)}]
\draw (-1,0)--(2,0);
\draw (0,-.5)--(0,3);
\draw  (-1.1,-.6)--(2.1,-.6)--(2.1,3.1)--(-1.1,3.1)--cycle;

\draw [->,thick] (0,0)--(1.25,-.25);
\draw [->,thick] (0,0)--(-.25,2.25);
\draw [->,thick,dashed,gray] (1.25,-.25)--(1,2);
\draw [->,thick,dashed,gray] (-.25,2.25)--(1,2);
\end{scope}

%chapter 5 pic 3
\begin{scope}[shift={(6.8,0)}]
\draw (-1,0)--(2,0);
\draw (0,-.5)--(0,3);
\draw  (-1.1,-.6)--(2.1,-.6)--(2.1,3.1)--(-1.1,3.1)--cycle;

\draw [->,thick] (0,0)--(-.5,.5);
\draw [->,thick] (0,0)--(1.5,1.5);
\fill (0,0) circle (.04); 
\fill (-.25,.25) circle (.04); 
\fill (.75,.75) circle (.04); 
\fill (.5,1) circle (.04); 
%\draw (.25,.5) circle (.55);

\draw [rotate around={45:(.25,.5)}] (.25,.5) ellipse (.57 and .5);

\end{scope}

\end{tikzpicture}
%
\else \if \thechapter A
%
\begin{tikzpicture}[>=latex]

%appendix pic 1 = chapter 2 pic 3
\begin{scope}
\draw (-1,0)--(2,0);
\draw (0,-.5)--(0,3);

\draw (-1,-.5)--(1.8,.9);
\draw [->,thick] (0,0)--(1,.5);
\draw [->,thick,dashed] (0,0)--(-.5,1.5);
\draw (-1,1.25)--(1.8,2.65);
\draw [->,thick] (-.5,1.5)--(.5,2);

\draw  (-1.1,-.6)--(2.1,-.6)--(2.1,3.1)--(-1.1,3.1)--cycle;
\end{scope}

%appendix pic 2 = chapter 3 pic 3
\begin{scope}[shift={(3.4,0)}]
\draw (-1,0)--(2,0);
\draw (0,-.5)--(0,3);
\draw  (-1.1,-.6)--(2.1,-.6)--(2.1,3.1)--(-1.1,3.1)--cycle;

\draw [->,thick] (0,0)--(1.25,-.25);
\draw [->,thick] (0,0)--(-.25,2.25);
\draw [->,thick,dashed,gray] (1.25,-.25)--(1,2);
\draw [->,thick,dashed,gray] (-.25,2.25)--(1,2);
\end{scope}

%appendix pic 3 = chapter 4 pic 3
\begin{scope}[shift={(6.8,0)}]
\draw (-1,0)--(2,0);
\draw (0,-.5)--(0,3);
\draw  (-1.1,-.6)--(2.1,-.6)--(2.1,3.1)--(-1.1,3.1)--cycle;

\draw [->,thick] (0,0)--(-.5,.5);
\draw [->,thick] (0,0)--(1.5,1.5);
\fill (0,0) circle (.04); 
\fill (-.25,.25) circle (.04); 
\fill (.75,.75) circle (.04); 
\fill (.5,1) circle (.04); 
%\draw (.25,.5) circle (.55);

\draw [rotate around={45:(.25,.5)}] (.25,.5) ellipse (.57 and .5);

\end{scope}

\end{tikzpicture}

\fi
\fi
\fi
\fi
\fi
\fi


    \reset@font\LARGE\scshape\bfseries\strut \textsc #1
    \par\vskip 10\p@
    \hrule height 1pt
    \vskip 50\p@
  }}
  
%\makeatletter
\def\@makesectionhead#1{%
	 {\reset@font\LARGE\itshape\bfseries\strut #1 \thechapter.\thesection \ #1
	 }}

\newcommand{\noin}{\noindent}
\newcommand{\ds}{\displaystyle}
\newcommand{\vs}[1]{\vskip #1in}
\newcommand{\tbf}[1]{\textbf{#1}}
\newcommand{\dds}{\displaystyle}
\newcommand{\bmx}[1]{\left[\hskip -3pt\begin{array}{#1} }
\newcommand{\emx}{\end{array}\hskip -3pt\right]}
\newcommand{\bdt}[1]{\left| \begin{array}{#1} }
\newcommand{\edt}{\end{array} \right|}

\newcommand{\btz}{\begin{center}\begin{tikzpicture}}
\newcommand{\etz}{\end{tikzpicture}\end{center}}

\newcommand{\vx}[1][]{\ensuremath{\vec{x_{#1}}}}
\newcommand{\vxp}{\ensuremath{\vec{x_p}}}
\newcommand{\vu}{\ensuremath{\vec{u}}}
\newcommand{\vv}{\ensuremath{\vec{v}}}
\newcommand{\vy}{\ensuremath{\vec{y}}}
\newcommand{\vz}{\ensuremath{\vec{z}}}
\newcommand{\vb}{\ensuremath{\vec{b}}}
\newcommand{\vw}{\ensuremath{\vec{w}}}
\newcommand{\veone}{\ensuremath{\vec{e_1}}}
\newcommand{\vetwo}{\ensuremath{\vec{e_2}}}
\newcommand{\vethree}{\ensuremath{\vec{e_3}}}
\newcommand{\vei}{\ensuremath{\vec{e_i}}}
\newcommand{\ven}[1]{\ensuremath{\vec{e_{#1}}}}
\newcommand{\rr}[1]{\ensuremath{\mathbb{R}^{#1}}}
\newcommand{\zero}{\ensuremath{\vec{\text{\it 0}}}}
\newcommand{\vect}[1]{\ensuremath{\vec{#1}}}
\newcommand{\arref}{\ensuremath{\overrightarrow{\text{rref}}}}
\newcommand{\vectt}[2]{\ensuremath{\bmx{c}#1\\#2\emx}}
\newcommand{\vecttt}[3]{\ensuremath{\bmx{c}#1\\#2\\#3\emx}}

\newcommand{\ttmm}{$M$}
\newcommand{\mm}{\texttt{M}}
\newcommand{\ii}[1]{\ensuremath{I_{#1}}}
\newcommand{\tta}{\ensuremath{A}}
\newcommand{\ttb}{\ensuremath{B}}
\newcommand{\ttc}{\ensuremath{C}}
\newcommand{\ttd}{\ensuremath{D}}
\newcommand{\ttm}{\ensuremath{M}}
\newcommand{\ttx}{\ensuremath{X}}
\newcommand{\tti}{\ensuremath{I}}
\newcommand{\tty}{\ensuremath{Y}}
\newcommand{\ttp}{\ensuremath{P}}
\newcommand{\ttat}{\ensuremath{A^T}}
\newcommand{\ttbt}{\ensuremath{B^T}}
\newcommand{\ttct}{\ensuremath{C^T}}
\newcommand{\ttdt}{\ensuremath{D^T}}
\newcommand{\ttmt}{\ensuremath{M^T}}
\newcommand{\ttxt}{\ensuremath{X^T}}
\newcommand{\ttit}{\ensuremath{I^T}}
\newcommand{\ttyt}{\ensuremath{Y^T}}
\newcommand{\ttai}{\ensuremath{A^{-1}}}
\newcommand{\ttbi}{\ensuremath{B^{-1}}}
\newcommand{\ttxi}{\ensuremath{X^{-1}}}
\newcommand{\ttpi}{\ensuremath{P^{-1}}}
\newcommand{\ttaxb}{\ensuremath{\tta\vx=\vb}}
\newcommand{\ttaxo}{\ensuremath{\tta\vx=\zero}}
\newcommand{\eyetwo}{\ensuremath{\bmx{cc}1&0\\0&1\emx}}
\newcommand{\eyethree}{\ensuremath{\bmx{ccc}1&0&0\\0&1&0\\0&0&1\emx}}
\newcommand{\eyefour}{\ensuremath{\bmx{cccc}1&0&0&0\\0&1&0&0\\0&0&1&0\\0&0&0&1\emx}}
\newcommand{\ma}{\ensuremath{A}}
\newcommand{\mb}{\ensuremath{B}}
\newcommand{\mc}{\ensuremath{C}}
\newcommand{\md}{\ensuremath{D}}
\newcommand{\tto}{\textbf{0}}
\renewcommand{\det}[1]{\text{det}\ensuremath{\left(#1\right)}}
\newcommand{\tr}{$^\text{\tt T}$}
\newcommand{\lda}{\ensuremath{\lambda}}
\newcommand{\rref}{reduced row echelon form}
\newcommand{\Rref}{Reduced row echelon form}
\newcommand{\el}{eigenvalue}
\newcommand{\ev}{eigenvector}
\newcommand{\realn}{\ensuremath{\mathbb{R}^n}}
\newcommand{\realm}{\ensuremath{\mathbb{R}^m}}
\newcommand{\realnm}{\ensuremath{\mathbb{R}^n\rightarrow\mathbb{R}^m}}
\newcommand{\real}[1]{\ensuremath{\mathbb{R}^{#1}}}
\newcommand{\rrr}[2]{\ensuremath{\mathbb{R}^{#1}\rightarrow\mathbb{R}^{#2}}}
\newcommand{\TT}{\ensuremath{[\, T\, ]}}

%%%% Begin Header TikZ

%  Some TiKZ  shortcuts to help make drawing 3D vectors faster.
%
\newcommand{\drawvect}[7]{\draw [#4] (0,0,0) -- (#1,0,0) -- (#1,#2,0) -- (#1,#2,#3);
  \draw [#5] (0,0,0) -- (#1,#2,#3) node [#6] {#7};}
\newcommand{\drawjustvect}[7]{\draw [#5] (0,0,0) -- (#1,#2,#3) node [#6] {#7};}

\newcommand{\drawxaxis}[4]{\draw [->] (#1,0,0) -- (#2,0,0) node [below left] {$x$};
\foreach \x in {#3,...,#4}
 {\draw (\x,-.1,0)--(\x,.1,0); } }
\newcommand{\drawyaxis}[4]{\draw [->] (0,#1,0) -- (0,#2,0) node [right] {$y$};
\foreach \y in {#3,...,#4}
 {\draw (0,\y,-.1)--(0,\y,.1); }; }
\newcommand{\drawzaxis}[4]{\draw [->] (0,0,#1) -- (0,0,#2) node [above] {$z$};
\foreach \z in {#3,...,#4}
 {\draw (0,-.1,\z)--(0,.1,\z); }; }
%
% 
% Draws the VMI Spider in TikZ
%
\newcommand{\vmispider}[1][]
{\begin{scope}[#1]
	\draw  (-2,2) -- (0,-1) -- (2,2);
	\draw  (-1,-1) -- (-1,2) -- (0,1) -- (1,2) -- (1,-1);
	\draw  (-1,2.5) -- (1,2.5);
	\draw  (-1,-1.5) -- (1,-1.5);
	\draw  (0,-1.5) -- (0,2.5);
\end{scope}
}
%
% Draws the unit square, easy to transform
%
\newcommand{\unitsquare}[1][]
{\begin{scope}
	\draw [#1] (0,0) node (A) {} -- (1,0) node (B) {} -- (1,1) node (C) {} -- (0,1) node (D) {} -- cycle;
	\draw [->,>=latex,#1,ultra thin] (0,.25)--(1,.25);
	\draw [->,>=latex,#1,ultra thin] (0,.5)--(1,.5);
	\draw [->,>=latex,#1,ultra thin] (0,.75)--(1,.75);
	\filldraw [black]  (A) circle (2pt);%black
	\filldraw [fill=white,thick]  (B) ++(-2pt,-2pt) rectangle ++(4pt,4pt);%red
	\filldraw [fill=white,thick]  (C) circle (2pt);%blue
	\filldraw [fill=white,thick]  (D) ++(-2.5pt,-2.5pt) -- ++(5pt,0pt) -- ++(-2.5pt,5pt) -- cycle;%green
	\end{scope}
}
%
% Draws unit square for cover image
%
\newcommand{\unitsquarecover}[1][]
{\begin{scope}
	\draw [#1] (0,0) node (A) {} -- (1,0) node (B) {} -- (1,1) node (C) {} -- (0,1) node (D) {} -- cycle;
	\draw [->,>=latex,#1,ultra thin] (0,.25)--(1,.25);
	\draw [->,>=latex,#1,ultra thin] (0,.5)--(1,.5);
	\draw [->,>=latex,#1,ultra thin] (0,.75)--(1,.75);
	\filldraw [black]  (A) circle (2pt);%black
	\draw [#1,ultra thick]  (B) ++(-2pt,-2pt) rectangle ++(4pt,4pt);%red
	\draw [ultra thick]  (C) circle (2pt);%blue
	\draw [#1,ultra thick]  (D) ++(-2.5pt,-2.5pt) -- ++(5pt,0pt) -- ++(-2.5pt,5pt) -- cycle;%green
	\end{scope}
}
%
% Draws unit square without the arrows in the middle.
%
\newcommand{\unitsquarewithoutarrows}[1][]
{\begin{scope}
	\draw [#1] (0,0) node (A) {} -- (1,0) node (B) {} -- (1,1) node (C) {} -- (0,1) node (D) {} -- cycle;
	\filldraw [black]  (A) circle (2pt);%black
	\filldraw [fill=white,thick]  (B) ++(-2pt,-2pt) rectangle ++(4pt,4pt);%red
	\filldraw [fill=white,thick]  (C) circle (2pt);%blue
	\filldraw [fill=white,thick]  (D) ++(-2.5pt,-2.5pt) -- ++(5pt,0pt) -- ++(-2.5pt,5pt) -- cycle;%green
	\end{scope}
}

%
% Draw x and y tick marks
%
\newcommand{\drawxticks}[1]
{\foreach \x in {#1}
		{\draw  (\x,-.1)--(\x,.1);
			};
}
\newcommand{\drawyticks}[1]
{\foreach \x in {#1}
		{\draw  (-.1,\x)--(.1,\x);
			};
}

\newcommand{\drawxlines}[3]
{\draw[<->] (#1,0) -- (#2,0) node [right] {$x$};
\foreach \x in {#3}
		{\draw  (\x,-.1)--(\x,.1);
			};
}

\newcommand{\drawylines}[3]
{\draw[<->] (0,#1) -- (0,#2) node [above] {$y$};
\foreach \x in {#3}
		{\draw  (-.1,\x)--(.1,\x);
			};
}

\newcommand{\asyouread}[1]{\begin{tikzpicture}
\ifthenelse{\boolean{in_color}}{\node [preaction={fill=black,opacity=.5,transform canvas={xshift=1mm,yshift=-1mm}}, right color=blue!80!black!30, left color=blue!80] at (0,0) {\textcolor{white}{\textsf{\textit{AS YOU READ $\ldots$}}}};}
{\node [preaction={fill=black,opacity=.5,transform canvas={xshift=1mm,yshift=-1mm}}, right color=black!30, left color=black!10] at (0,0) {\textcolor{white}{\textsf{\textit{AS YOU READ $\ldots$}}}};}
\end{tikzpicture}
\begin{enumerate}
#1
\end{enumerate}
\vskip 20pt}

%%%%
%%  A new figure environment, trying to fix the float problem.
%%
%%%%

\newcounter{myfigurecounter}[chapter]
\renewcommand\themyfigurecounter{\thechapter.\arabic{myfigurecounter}}
\newenvironment{myfigure}{\refstepcounter{myfigurecounter}}{}
\newcommand{\mycaption}[1]{%
\begin{center}%
\vskip -1.5\baselineskip
\begin{tikzpicture}%
\draw (0,0) node [text width=\textwidth,align=center] {Figure \themyfigurecounter: #1};%
\end{tikzpicture}%
\end{center}%
}