\documentclass[10pt]{article}

\input{Header_APEX}
%\usepackage{underscore}
\usepackage{lipsum}


\begin{document}
A first example using \verb|\example|. \\

\example{ex_first}{\lipsum[2]}{\lipsum[2]}\\

Now repeat in black and white by using \verb|\printinblackandwhite|.

Also, note how the line at the end above extends down ``too far''. Fix this by adding a \verb|\vskip -\baselineskip| at the end. (This is due to my use of the \verb|lipsum| package for ``dummy'' text.)\\

\printinblackandwhite

\example{ex_second}{\lipsum[2]}{\lipsum[2]\vskip -\baselineskip}\\

Now use \verb|\setcolorlinecolor{red}| to make the color red; also, by default, changes the printing back to color printing. \\

\setcolorlinecolor{red}
\example{ex_third}{\lipsum[2]}{\lipsum[4]\vskip -\baselineskip}\\

Change the style using \verb|\setlinestyle{thin,dashed}|.

We also use the \verb|\example*| version that does not include {\bf Solution.}\\
\setlinestyle{thin,dashed}

\example*{ex_fourth}{\lipsum[2]\vskip -\baselineskip}

We now repeat all of this with {\bf Example} in the margin by using the 

\noindent\verb|\exampleinmargin| command. 

We also change the line back to its default setting first with 

\noindent\verb|\setcolorlinecolor{blue!95!black!30}| and \verb|\setlinestyle{thick}|.\\

\setcolorlinecolor{blue!95!black!30}
\setlinestyle{thick}
\exampleinmargin

\example{ex_fifth}{\lipsum[1]}{\lipsum[2]\vskip -\baselineskip}\\

\printinblackandwhite

\example{ex_six}{\lipsum[3]}{\lipsum[2]\vskip -\baselineskip}\\

\setcolorlinecolor{red}
\example{ex_teve}{\lipsum[1]}{\lipsum[4]\vskip -\baselineskip}\\

\setlinestyle{thin,dashed}
\example*{ex_eith}{\lipsum[1]\vskip -\baselineskip}



\end{document}