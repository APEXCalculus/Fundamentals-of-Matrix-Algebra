\newcounter{examplecounter}
\newcounter{definitioncounter}
\newcounter{theoremcounter}
\newcounter{ideacounter}
\newcounter{excounter}

\definecolor{myblue}{rgb}{.7,.7,1}
\definecolor{mygreen}{rgb}{.7,1,.7}
\definecolor{myred}{rgb}{1,.7,.7}
\definecolor{myyellow}{rgb}{.9,.9,0}

\newtheoremstyle{mystyle}% Generic Theorem Style
  {15pt}%      Space above
  {15pt}%      Space below
  {\sffamily}%         Body font
  {}%         Indent amount (empty = no indent, \parindent = para indent)
  {\sffamily\bfseries}% Thm head font
  {}%        Punctuation after thm head
  {1.5em}%     Space after thm head: " " = normal interword space;
        %       \newline = linebreak
  {}%         Thm head spec (can be left empty, meaning `normal')

\newtheoremstyle{myexamplestyle}% Used for the Example environments
  {15pt}%      Space above
  {5pt}%      Space below
  {}%         Body font
  {}%         Indent amount (empty = no indent, \parindent = para indent)
  {\bfseries}% Thm head font
  {}%        Punctuation after thm head
  {1.5em}%     Space after thm head: " " = normal interword space;
        %       \newline = linebreak
  {}%         Thm head spec (can be left empty, meaning `normal')


%
%Define style for the Examples  %%%%%%%%%%%%%%%%%%%%%%%%%%%%%%%%%%%%%%%%%%%%%%%%%%%%%%%%%%%%%%%%%%%%%%%
%%%%%%%%%%%%%%%%%%%%%%%%%%%%%%%%%%%%%%%%%%%%%%%%%%%%%%%%%%%%%%%%%%%%%%%%%%%%%%%%%%%%%%%%%%%%%%%%%%%%%%%
%
\theoremstyle{myexamplestyle}

\newtheorem{myexample}{Example}



\newboolean{showexamplelines}
\setboolean{showexamplelines}{true}
%\setboolean{showexamplelines}{false}

%%%
%%%
\ifthenelse{\boolean{booksize}}{%booksize example environment
\newcommand{\example}[3]{%
\hskip -50pt%
\begin{tikzpicture}[remember picture,overlay]
\draw (0,0) node [yshift=2.5pt]
(#1)% 
[rectangle, text width = 100pt,text justified]%
{\begin{myexample}\label{#1}\end{myexample}};
\end{tikzpicture}%
\hskip 50pt%
\ifthenelse{\boolean{showexamplelines}}% Should we show lines or not?
{% Begin if to show lines
\ifthenelse{\pageref{#1} = \pageref{e#1}}
{}% If on the same page, do nothing from the top
{% if on different pages - assuming the next page
\ifthenelse{\isodd{\pageref{#1}}}%
{% if an odd page
\begin{tikzpicture}[remember picture, overlay]
\draw [\ifthenelse{\boolean{in_color}}{blue!95!black!30}{black!30},line width=1pt] 
(#1.south west) -- (oddpagebottom |- #1.south west) -- %
(oddpagebottom);
\end{tikzpicture}%
}% close if an odd page
%\else
{% else if on an even page
\begin{tikzpicture}[remember picture, overlay]
\draw [\ifthenelse{\boolean{in_color}}{blue!95!black!30}{black!30},line width=1pt] 
(#1.south west) -- (evenpagebottom |- #1.south west) -- %
(evenpagebottom);
\end{tikzpicture}%
}%\fi%close if on an even page
}%close if on different pages
%closes ifthenelse
#2%
\vskip \baselineskip%
\hskip -80pt%
\parbox{80pt}{\textsc{\small Solution}}%
#3\label{e#1}%
\ifthenelse{\pageref{#1} = \pageref{e#1}}
{% if on the same page
\ifthenelse{\isodd{\pageref{e#1}}}
{% if an odd page
\begin{tikzpicture}[remember picture, overlay]
\node (e#1) {};
\draw [\ifthenelse{\boolean{in_color}}{blue!95!black!30}{black!30},line width=1pt]  
(#1.south west) -- (oddpagebottom |- #1.south west) -- (oddpagebottom |- e#1.south west)  -- ++(0,-.1) -- ++(1,0);
\end{tikzpicture}%
}% close if an odd page
%\else
{% else if on an even page
\begin{tikzpicture}[remember picture, overlay]
\node (e#1) {};
\draw [\ifthenelse{\boolean{in_color}}{blue!95!black!30}{black!30},line width=1pt] 
(#1.south west) -- (evenpagebottom |- #1.south west) -- %
(evenpagebottom |- e#1.south west) -- ++(0,-.1) -- ++(1,0);
\end{tikzpicture}%
}%\fi%close if on an even page
}%close if on the same page
{% if on different pages - assuming the next page
%\ifodd\thepage
\ifthenelse{\isodd{\pageref{e#1}}}
{% if an odd page
\begin{tikzpicture}[remember picture, overlay]
\node (e#1) {};
\draw [\ifthenelse{\boolean{in_color}}{blue!95!black!30}{black!30},line width=1pt] 
(oddpagetop) -- (oddpagetop |- e#1.south west) -- ++(0,-.1) -- ++(1,0);
\end{tikzpicture}%
}% close if an odd page
%\else
{% else if on an even page
\begin{tikzpicture}[remember picture, overlay]
\node (e#1) {};
\draw [\ifthenelse{\boolean{in_color}}{blue!95!black!30}{black!30},line width=1pt] 
(evenpagetop) -- (evenpagetop |- e#1.south west) -- ++(0,-.1) -- ++(1,0);
\end{tikzpicture}%
}%\fi%close if on an even page
}%close if on different pages
%closes ifthenelse
}% Closes If to show example lines
{% Else - do not show lines
#2%
\vskip \baselineskip%
\hskip -80pt%
\parbox{80pt}{\textsc{\small Solution}}%
#3\label{e#1}%
}% Ends Else - do not show lines
\ \\ }%ends definition
}%%%%  ends booksize example environment
{%
\newcommand{\example}[3]{% begins ipod and laptop size environment
%\noindent
{\begin{tikzpicture}[remember picture, overlay]
\node [xshift=15pt,yshift=2.5pt] (#1) [rectangle,
text width = 70pt,
text justified,fill=blue!95!black!30] 
{\phantom{Example}};
\end{tikzpicture}
\begin{tikzpicture}[remember picture, overlay]
%\node [yshift=2.5pt] (#1) [rectangle,
%text width = 150pt,
%text justified] 
\node [xshift=53pt,yshift=2.5pt] (#1) [rectangle,
text width = 150pt,
text justified,text=red] 
{\begin{myexample}\label{#1}\end{myexample}};
\end{tikzpicture} {\hskip 50pt}{\hskip 4pt}}
\ifthenelse{\boolean{showexamplelines}}% Should we show lines or not?
{ % Begin if to show lines
\ifthenelse{\pageref{#1} = \pageref{e#1}}
{} % If on the same page, do nothing from the top
{% if on different pages - assuming the next page
\ifthenelse{\isodd{\pageref{#1}}}
		{% if an odd page
		\begin{tikzpicture}[remember picture, overlay]
		\draw [red!95!black!30,line width=1pt] 
		(oddpagebottom |- #1.south west) -- %
		(oddpagebottom);
		\end{tikzpicture}%
		}% close if an odd page
		%\else
		{% else if on an even page
		\begin{tikzpicture}[remember picture, overlay]
		\draw [red!95!black!30,line width=1pt] 
	  (evenpagebottom |- #1.south west) -- %
		(evenpagebottom);
		\end{tikzpicture}%
		}%\fi%close if on an even page
} %close if on different pages
%closes ifthenelse
\nopagebreak\noindent {\hskip -5pt} #2
\vskip .2in {\hskip -80pt \parbox{72pt}{{\textsc{\small Solution}}}}
\noindent #3 \label{e#1}  % draw lines from bottom
\ifthenelse{\pageref{#1} = \pageref{e#1}}
{% if on the same page
\ifthenelse{\isodd{\pageref{e#1}}}
		{% if an odd page
		\begin{tikzpicture}[remember picture, overlay]
		\node (e#1) {};
		%\draw (#1) -- (e#1);
		\draw [red!95!black!30,line width=1pt]  
		(#1.south west) -- (oddpagebottom |- #1.south west) -- (oddpagebottom |- e#1.south west)  -- ++(0,-.1) -- ++(1,0);
		\end{tikzpicture}%
		}% close if an odd page
		%\else
		{% else if on an even page
		\begin{tikzpicture}[remember picture, overlay]
		\node (e#1) {};
		\draw [red!95!black!30,line width=1pt] 
		(#1.south west) -- (evenpagebottom |- #1.south west) -- %
		(evenpagebottom |- e#1.south west) -- ++(0,-.1) -- ++(1,0);
		\end{tikzpicture}%
		}%\fi%close if on an even page
} %close if on the same page
{% if on different pages - assuming the next page
%\ifodd\thepage
\ifthenelse{\isodd{\pageref{e#1}}}
		{% if an odd page
		\begin{tikzpicture}[remember picture, overlay]
		\node (e#1) {};
		\draw [red!95!black!30,line width=1pt] 
		(oddpagetop) -- (oddpagetop |- e#1.south west) -- ++(0,-.1) -- ++(1,0);
		\end{tikzpicture}%
		}% close if an odd page
		%\else
		{% else if on an even page
		\begin{tikzpicture}[remember picture, overlay]
		\node (e#1) {};
		\draw [red!95!black!30,line width=1pt] 
		(evenpagetop) -- (evenpagetop |- e#1.south west) -- ++(0,-.1) -- ++(1,0);
		\end{tikzpicture}%
		}%\fi%close if on an even page
} %close if on different pages
%closes ifthenelse
} % Closes If to show example lines
{ % Else - do not show lines
\nopagebreak\noindent {\hskip -5pt} #2
\vskip .2in {\hskip -80pt \parbox{72pt}{{\textsc{\small Solution}}}}
\noindent #3 \label{e#1} 
} % Ends Else - do not show lines
\ \\ }%ends definition
}

%%%%
%% Defines the examples for the CreateSpace sized book

\ifthenelse{\boolean{amazonsize}}{%booksize example environment
\renewcommand{\example}[3]{%
%\hskip -10pt%
{\begin{tikzpicture}[remember picture, overlay]
\node [xshift=15pt,yshift=2.5pt] (#1) [rectangle,
text width = 70pt,
text justified] 
{\phantom{Example}};
\end{tikzpicture}
\begin{tikzpicture}[remember picture, overlay]
%\node [yshift=2.5pt] (#1) [rectangle,
%text width = 150pt,
%text justified] 
\node [xshift=58pt,yshift=2.5pt] (#1) [rectangle,
text width = 150pt,
text justified] 
{\begin{myexample}\label{#1}\end{myexample}};
\end{tikzpicture}{\hskip 74pt}%{\hskip 4pt}%
}
\ifthenelse{\boolean{showexamplelines}}% Should we show lines or not?
{% Begin if to show lines
\ifthenelse{\pageref{#1} = \pageref{e#1}}
{}% If on the same page, do nothing from the top
{% if on different pages - assuming the next page
\ifthenelse{\isodd{\pageref{#1}}}%
{% if an odd page
\begin{tikzpicture}[remember picture, overlay]
\ifthenelse{\boolean{in_color}}{\draw [blue!95!black!30,line width=1pt]}{\draw [black!30,line width=1pt]} 
(oddpagebottom |- #1.north west) -- %
(oddpagebottom);
\end{tikzpicture}%
}% close if an odd page
%\else
{% else if on an even page
\begin{tikzpicture}[remember picture, overlay]
\ifthenelse{\boolean{in_color}}{\draw [blue!95!black!30,line width=1pt]}{\draw [black!30,line width=1pt]} 
(evenpagebottom |- #1.north west) -- %
(evenpagebottom);
\end{tikzpicture}%
}%\fi%close if on an even page
}%close if on different pages
%closes ifthenelse
#2%
\vskip \baselineskip%
%\hskip -10pt%
\parbox{80pt}{\textsc{\small \bfseries Solution}}%
#3\label{e#1}%
\ifthenelse{\pageref{#1} = \pageref{e#1}}
{% if on the same page
\ifthenelse{\isodd{\pageref{e#1}}}
{% if an odd page
\begin{tikzpicture}[remember picture, overlay]
\node (e#1) {};
\ifthenelse{\boolean{in_color}}{\draw [blue!95!black!30,line width=1pt]}{\draw [black!30,line width=1pt]} 
(oddpagebottom |- #1.north west) -- (oddpagebottom |- e#1.north west)  -- ++(0,-.2) -- ++(1,0);
\end{tikzpicture}%
}% close if an odd page
%\else
{% else if on an even page
\begin{tikzpicture}[remember picture, overlay]
\node (e#1) {};
\ifthenelse{\boolean{in_color}}{\draw [blue!95!black!30,line width=1pt]}{\draw [black!30,line width=1pt]}
(evenpagebottom |- #1.north west) -- %
(evenpagebottom |- e#1.north west) -- ++(0,-.2) -- ++(1,0);
\end{tikzpicture}%
}%\fi%close if on an even page
}%close if on the same page
{% if on different pages - assuming the next page
%\ifodd\thepage
\ifthenelse{\isodd{\pageref{e#1}}}
{% if an odd page
\begin{tikzpicture}[remember picture, overlay]
\node (e#1) {};
\ifthenelse{\boolean{in_color}}{\draw [blue!95!black!30,line width=1pt]}{\draw [black!30,line width=1pt]}
(oddpagetop) -- (oddpagetop |- e#1.south west) -- ++(0,-.2) -- ++(1,0);
\end{tikzpicture}%
}% close if an odd page
%\else
{% else if on an even page
\begin{tikzpicture}[remember picture, overlay]
\node (e#1) {};
\ifthenelse{\boolean{in_color}}{\draw [blue!95!black!30,line width=1pt]}{\draw [black!30,line width=1pt]} 
(evenpagetop) -- (evenpagetop |- e#1.south west) -- ++(0,-.2) -- ++(1,0);
\end{tikzpicture}%
}%\fi%close if on an even page
}%close if on different pages
%closes ifthenelse
}% Closes If to show example lines
{% Else - do not show lines
#2%
\vskip \baselineskip%
%\hskip -10pt%
\parbox{80pt}{\textsc{\small\bfseries Solution}}%
#3\label{e#1}%
}% Ends Else - do not show lines
\ \\ }%ends definition
}
{}

%%%%


%%%
%%%

%	Draws an example line when the beginning and end of the example
%	spans 3 or more pages. This assumes it is a total of 3 pages;
% this calculates the line to be drawn as one page after the start, 
% calculating even/odd page side this way.
%
\newcommand{\drawexampleline}[1]{
\ifthenelse{\boolean{showexamplelines}}
{ % show lines
\ifthenelse{\isodd{\pageref{#1}}}
	{\begin{tikzpicture}[remember picture,overlay]
	 \ifthenelse{\boolean{in_color}}{\draw [blue!95!black!30,line width=1pt]}{\draw [black!30,line width=1pt]}
	 (evenpagetop) -- (evenpagebottom);
	 \end{tikzpicture}
}
	{\begin{tikzpicture}[remember picture,overlay]
	 \ifthenelse{\boolean{in_color}}{\draw [blue!95!black!30,line width=1pt]}{\draw [black!30,line width=1pt]}
	 (oddpagetop) -- (oddpagebottom);
	 \end{tikzpicture}
}
} % End show lines
{} % No else - don't do anything
}

%  This draws an example line on the current page.
%  It takes a number as an argument (it could be the page number).
%  Use an odd number for an odd page, even for an even page.
%
\newcommand{\drawexamplelinenow}[1]{
\ifthenelse{\boolean{showexamplelines}}
{ % show lines
\ifthenelse{\isodd{#1}}
	{\begin{tikzpicture}[remember picture,overlay]
	 \ifthenelse{\boolean{in_color}}{\draw [blue!95!black!30,line width=1pt]}{\draw [black!30,line width=1pt]}
	  (oddpagetop) -- (oddpagebottom);
	 \end{tikzpicture}
}
	{\begin{tikzpicture}[remember picture,overlay]
	 \ifthenelse{\boolean{in_color}}{\draw [blue!95!black!30,line width=1pt]}{\draw [black!30,line width=1pt]}
	  (evenpagetop) -- (evenpagebottom);
	 \end{tikzpicture}
}
} % End show lines
{} % No else - don't do anything
}





%
%Define style for Definitions, Theorems and Key Ideas  %%%%%%%%%%%%%%%%%%%%%%%%%%%%%%%%%%%%%%%%%%%%
%%%%%%%%%%%%%%%%%%%%%%%%%%%%%%%%%%%%%%%%%%%%%%%%%%%%%%%%%%%%%%%%%%%%%%%%%%%%%%%%%%%%%%%%%%%%%%%%%%%
%

\theoremstyle{mystyle}

\newtheorem{deff}{Definition}

%\newcommand{\definition}[2]{\begin{deff}\label{#1} \fcolorbox{blue}{myblue}{ \begin{minipage}[t]{265pt}#2\end{minipage}}\end{deff}}

\ifthenelse{\boolean{booksize}\or\boolean{amazonsize}}{% Begin if booksize
\newcommand{\definition}[2]{\begin{deff}\label{#1}
\raisebox{24pt}{\begin{tikzpicture}[baseline=(current bounding box.north)-1cm]
%\pgfsetbaselinepointlater{\pgfpointanchor{X}{base}}
\ifthenelse{\boolean{in_color}}{\node (X) [rectangle,
text width = 240pt,
baseline=-1cm,
very thick,
inner sep=15pt,
draw = yellow!95!black!60,
top color = white!95!yellow,
bottom color = yellow!90!black!30,
text justified]} 
{\node (X) [rectangle,draw,
text width = 240pt,
baseline=-1cm,
inner sep=15pt,
very thick,
text justified]} 
{#2};
\end{tikzpicture}}
\end{deff}}%
}%  End If
{% begin else not booksize
\newcommand{\definition}[2]{\begin{deff}\label{#1}
\raisebox{24pt}{\begin{tikzpicture}[baseline=(current bounding box.north)-1cm]
%\pgfsetbaselinepointlater{\pgfpointanchor{X}{base}}
\node (X) [rectangle,draw,
text width = 170pt,
baseline=-1cm,
inner sep=15pt,
very thick,
text justified] 
{#2};
\end{tikzpicture}}
\end{deff}}%
}%

%%%%%%%%%%%%%%%%%%%%%%%%%%%%%%%%%%%%%%%%%%%%%%%%%%%%%%%%%%%%%%%%%%%%%%%%%%%%%%%%%%%%%%%%
%%%%%%%%%%%%%%%%%%%%%%%%%%%%%%%%%%%%%%%%%%%%%%%%%%%%%%%%%%%%%%%%%%%%%%%%%%%%%%%%%%%%%%%%

\newtheorem{thm}{Theorem}

\ifthenelse{\boolean{booksize}\or\boolean{amazonsize}}{%
\newcommand{\colortheorem}[2]{\begin{thm}\label{#1}
\raisebox{24pt}{\begin{tikzpicture}[baseline=(current bounding box.north)-1cm]
%\pgfsetbaselinepointlater{\pgfpointanchor{X}{base}}
\ifthenelse{\boolean{in_color}}
{\node (X) [rectangle,
text width = 240pt,
baseline=-1cm,
inner sep=15pt,
very thick,
draw = green!30!black!50,
top color = white!95!green,
bottom color = green!60!black!20,
text justified] }
{\node (X) [rectangle,draw,
text width = 240pt,
baseline=-1cm,
inner sep=15pt,
very thick,
text justified] }
{#2};
\end{tikzpicture}}\end{thm}}%
}%
{%
\newcommand{\colortheorem}[2]{\begin{thm}\label{#1}
\raisebox{24pt}{\begin{tikzpicture}[baseline=(current bounding box.north)-1cm]
%\pgfsetbaselinepointlater{\pgfpointanchor{X}{base}}
\node (X) [rectangle,draw,
text width = 170pt,
baseline=-1cm,
inner sep=15pt,
very thick,
text justified] 
{#2};
\end{tikzpicture}}\end{thm}}%
}%

%%%%%%%%%%%%%%%%%%%%%%%%%%%%%%%%%%%%%%%%%%%%%%%%%%%%%%%%%%%%%%%%%%%%%%%%%%%%%%%%%%%%%%%%
%%%%%%%%%%%%%%%%%%%%%%%%%%%%%%%%%%%%%%%%%%%%%%%%%%%%%%%%%%%%%%%%%%%%%%%%%%%%%%%%%%%%%%%%

\newtheorem{keyidea}{Key Idea}

\ifthenelse{\boolean{booksize}\or\boolean{amazonsize}}{%
\newcommand{\idea}[2]{\begin{keyidea}\label{#1}
\raisebox{24pt}{\begin{tikzpicture}[baseline=(current bounding box.north)-1cm]
%\pgfsetbaselinepointlater{\pgfpointanchor{X}{base}}
\ifthenelse{\boolean{in_color}}
{\node (X) [rectangle,
text width = 240pt,
baseline=-1cm,
inner sep=15pt,
very thick,
draw = red!30!black!50,
top color = white!95!red,
bottom color = red!60!black!20,
text justified] }
{\node (X) [rectangle,draw,
text width = 240pt,
baseline=-1cm,
inner sep=15pt,
very thick,
text justified] }
{#2};
\end{tikzpicture}}
\end{keyidea}}%
}%
{%
\newcommand{\idea}[2]{\begin{keyidea}\label{#1}
\raisebox{24pt}{\begin{tikzpicture}[baseline=(current bounding box.north)-1cm]
%\pgfsetbaselinepointlater{\pgfpointanchor{X}{base}}
\node (X) [rectangle,draw,
text width = 170pt,
baseline=-1cm,
inner sep=15pt,
very thick,
text justified] 
{#2};
\end{tikzpicture}}
\end{keyidea}}%
}%

%%%%%%%%%%%%%%%%%%%%%%%%%%%%%%%%%%%%%%%%%%%%%%%%%%%%%%%%%%%%%%%%%%%%%%%%%%%%%%%%%%%%%%%%
%%%%%%%%%%%%%%%%%%%%%%%%%%%%%%%%%%%%%%%%%%%%%%%%%%%%%%%%%%%%%%%%%%%%%%%%%%%%%%%%%%%%%%%%

\newcommand{\asyouread}[1]{\begin{tikzpicture}
\ifthenelse{\boolean{in_color}}{\node [preaction={fill=black,opacity=.5,transform canvas={xshift=1mm,yshift=-1mm}}, right color=blue!80!black!30, left color=blue!80] at (0,0) {\textcolor{white}{\textsf{\textit{AS YOU READ $\ldots$}}}};}
{\node [preaction={fill=black,opacity=.5,transform canvas={xshift=1mm,yshift=-1mm}}, right color=black!30, left color=black!10] at (0,0) {\textcolor{white}{\textsf{\textit{AS YOU READ $\ldots$}}}};}
\end{tikzpicture}
\begin{enumerate}
#1
\end{enumerate}
\vskip 20pt}


