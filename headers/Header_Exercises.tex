%
%Commands used in the Exercise section %%%%%%%%%%%%%%%%%%%%%%%%%%%%%%%%%%%%%%%%%%%%%%%%%%%%%%%%%%
%%%%%%%%%%%%%%%%%%%%%%%%%%%%%%%%%%%%%%%%%%%%%%%%%%%%%%%%%%%%%%%%%%%%%%%%%%%%%%%%%%%%%%%%%%%%%%%%

\newcommand{\exc}{\addtocounter{excounter}{2}\arabic{excounter}}

%\newcount\edc  %exercise display count

\newif\ifmore

\newif\ifexsetmore

\newcount\showexercises

\newcount\numberofexercises

\newcounter{numofexer}
\newcounter{negnumofexer}

\newcounter{debug}
\setcounter{debug}{0}

\newcounter{exercisecounter}
\newcounter{IMTcount}
\newcounter{IMTcount_temp}

\newboolean{showexercisenames}
%\setboolean{showexercisenames}{true}
\setboolean{showexercisenames}{false}

\newread\exread
\newread\exsetread
\newread\exansread
\newread\printansread

\newwrite\answrite
\openout\answrite=\jobname.answers

\def\exinput #1 {\ifnum \showexercises=1 
												\openin\exread=#1 
												\read\exread to \tempp 
												\begin{enumerate} 
													\addtocounter{enumi}{\theexercisecounter}
													\item 
													\ifthenelse{\boolean{showexercisenames}}
													{\tiny {\hskip -60pt}% This line too
												  \makebox[60pt][l]{\printexercisename #1 }%  
												  \small%
												  }{}
													\tempp 
													\addtocounter{exercisecounter}{1}
												\end{enumerate}
												\closein\exread 
									\else \ifnum \showexercises=2  %i.e, we are printing odd answers, not questions 
												\openin\exread=#1 
												\read\exread to \tempp % read in the question - we ignore it.
												\addtocounter{exercisecounter}{1}
												\read\exread to \tempp % reads in the answer
												\ifodd \theexercisecounter
												%\else
													\begin{enumerate} 
													\addtocounter{enumi}{\theexercisecounter}
													\addtocounter{enumi}{-1}
													\item 
													\ifthenelse{\boolean{showexercisenames}}
													{\tiny {\hskip -60pt}% This line too
												  \makebox[60pt][l]{\printexercisename #1 }%
												  \small%
												  }{}
													\tempp 
													%\addtocounter{exercisecounter}{1}
													\end{enumerate} 
												\fi
												\closein\exread 
												\fi  %ends the \ifnum \showexercises = 2 if statement
												%
												\ifnum \showexercises=3  %i.e., we print all answers, not just odds 
												\openin\exread=#1 
												\read\exread to \tempp %reads in the question, which is ignored 
												\read\exread to \tempp %reads in the answer
												\begin{enumerate} 
													\addtocounter{enumi}{\theexercisecounter}
													\item 
													\ifthenelse{\boolean{showexercisenames}}
													{\tiny {\hskip -60pt}% This line too
												  \makebox[60pt][l]{\printexercisename #1 }%
												  \small%
												  }{}
													\tempp 
													\addtocounter{exercisecounter}{1}
												\end{enumerate}
												\closein\exread
												\fi % ends the \ifnum \showexercises=3 if statement
									\fi 
								}

\def\exsetinput #1 {\openin\exsetread=#1
										\setcounter{numofexer}{0}
										\setcounter{negnumofexer}{0} 
										\read\exsetread to \exsettemp
										\read\exsetread to \exsettemp
										{\loop
												\read\exsetread to \exsettemp
												\ifeof \exsetread \exsetmorefalse \else \exsetmoretrue \fi
												\ifexsetmore
														\addtocounter{numofexer}{1}
														\addtocounter{negnumofexer}{-1}
											\repeat}							
										\closein\exsetread
										\openin\exsetread=#1
										\ifnum \showexercises=1 
											\read\exsetread to \exsettemp
											\setcounter{enumi}{\theexercisecounter} 
											\addtocounter{enumi}{1}
											\ifthenelse{\boolean{showexercisenames}}
													{\tiny {\hskip -60pt}% This line too
												  \makebox[60pt][l]{\printexercisename #1 }%
												  \small%
												  }{}
											\noin\textbf{\exsettemp\theenumi\addtocounter{enumi}{-1}
											\addtocounter{enumi}{\thenumofexer}{-- }\theenumi%
											\addtocounter{enumi}{\thenegnumofexer}%
											\read\exsetread to \exsettemp \exsettemp}%
											
											{\loop
													\read\exsetread to \exsettemp
													\ifeof \exsetread \exsetmorefalse \else \exsetmoretrue \fi
													\ifexsetmore
															\exsettemp
											\repeat}
										\else
											\read\exsetread to \exsettemp
											\read\exsetread to \exsettemp
											{\loop
													\read\exsetread to \exsettemp
													\ifeof \exsetread \exsetmorefalse \else \exsetmoretrue \fi
													\ifexsetmore
															\exsettemp
											\repeat}
										\fi
										\closein\exsetread
								}

\def\printexercises #1 {%
\ifthenelse{\equal{\expandafter\readsection\thesection}{1}}{\immediate\write\answrite{\noexpand \noindent {\noexpand\Large\noexpand\bf Chapter \thechapter} \noexpand \vskip \noexpand\baselineskip } \write\answrite{}}{}%
\immediate\write\answrite{\noexpand\noindent {\noexpand\bf Section \thesection} \noexpand \vskip \baselineskip}%
\write\answrite{\noexpand\printanswers{#1}}%
\setcounter{exercisecounter}{0}\showexercises=1 \small%
\noin\underline{\parbox{\textwidth}{\Large\textbf{Exercises \thesection} }}% 
\sffamily%
\vskip\baselineskip%
\begin{multicols}{2}%
				\openin\exansread=#1 
				\ifeof \exansread 
					{No problems written.} 
				\else 
					\loop \read\exansread to \extemp  
							\ifeof \exansread \morefalse \else \moretrue \fi 
							\ifmore 
									\extemp
							\repeat 
				\fi 
				\closein\exansread 
				\end{multicols}%
				\setlength{\hoffset}{0pt} \rmfamily\normalsize \vskip \baselineskip% \end{flushleft}
				}
				


% The following prints the answers. In order to print just odds, set \showexercises=2. 
% To print all answers, set \showexercises=3.
% 
\def\printanswers #1 {\setcounter{exercisecounter}{0} \footnotesize \showexercises=2 \openin\printansread=#1 
				\ifeof \printansread 
					{No problems written.} 
				\else 
					\loop \read\printansread to \extemp  
							\ifeof \printansread \morefalse \else \moretrue \fi 
							\ifmore 
									\extemp
							\fi 
							\ifeof \printansread \morefalse \else \moretrue \fi 
							\ifmore 
					\repeat 
				\fi 
				\closein\printansread
				\small}

\def\testexinput #1 {\ifnum \showexercises=1 
												\openin\exread=#1 
												\read\exread to \tempp 
												\begin{enumerate} 
													\addtocounter{enumi}{\theexercisecounter}
													\item \tempp 
													\addtocounter{exercisecounter}{1}
												\end{enumerate}
												\closein\exread 
									\else \ifnum \showexercises=2 
												\openin\exread=#1 
												\read\exread to \tempp 
												\addtocounter{exercisecounter}{1}
												\read\exread to \tempp 
												\ifodd \theexercisecounter
														\begin{enumerate} 
													\addtocounter{enumi}{\theexercisecounter}
													\addtocounter{enumi}{-1}
													\item \tempp 
													%\addtocounter{exercisecounter}{1}
													\end{enumerate} 
												\fi
												\closein\exread 
												\fi
									\fi 
								}
								
\def \printexercisename exercises/#1_#2_#3_#4 {#1 #2 #3 #4}

%%%%%%%%%%%%% Used to automate the answer production at
%%%%%%%%%%%%% end the text.

\def \readsection #1.#2{#2}

\def \writeexercisestofile #1{%
\write\answrite{\noexpand\printanswers{exercises/0\thechapter_0\expandafter\readsection #1_exercises.tex} \noexpand \vskip \baselineskip } \write\answrite{} }

\def \cchapter #1{\chapter{#1}\write\answrite{\noexpand \noindent {\Large \bf Chapter \thechapter} \noexpand \vskip \noexpand\baselineskip } \write\answrite{} }

\def \ssection #1{\section{#1}\write\answrite{\noexpand\noindent {\bf Section \thesection}} \noexpand \vskip \baselineskip
}

\def \makeexercisesection #1{\write\answrite{\noexpand\end{multicols}
\noexpand\normalsize} \closeout\answrite \chapter{#1} \input{\jobname.answers}}%


%
% The following is a line of code, not a definition. 
% It writes the first line of the ``.answers'' file
% to set up the proper formatting of that file.
%
\write\answrite{\noexpand\small\noexpand\raggedright\noexpand\begin{multicols}{2}}
