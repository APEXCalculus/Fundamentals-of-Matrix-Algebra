1.1

Annoying habits
Difference between constants and coeff.
Can a coefficient be 0?

1.2

What is remarkable about the definition of a matrix?
Vertical lines of numbers in a matrix are called what?
In a matrix \tta, the entry a_{53} refers to which entry?

1.3
Give two reasons why the Elementary Row Operations are called ``Elementary.''
Assuming a solution exists, can all linear systems of equations be solved using only elementary row operations?
Give one reason why one might not be interested in putting a matrix into \rref.
Identify the leading 1s in the following matrix:
	$$\bmx{cccc} 1&0&0&1\\ 0&1&1&0\\0&0&1&1\\0&0&0&0\emx$$
Using the ``forward'' and ``backward'' steps of Gaussian elimination create lots of \underline{\hskip .5in} making computations easier.

1.4
Can a system of two equations and two unknowns have no solution?
T/F: It possible for a linear system to have exactly 5 solutions.
T/F: A variable that corresponds to a leading 1 is ``free.''
T/F: If a linear system implies that $0=1$, then it has no solution.
T/F: A particular system for a linear system with infinite solutions can be found by arbitrarily picking values for the free variables.

1.5
How do most problems appear ``in the real world?''
The unknowns in a problem are also called what?
How many points are needed to determine the coefficients of a 5$^\text{th}$ degree polynomial?

2.1
When are two matrices equal?
Write an explanation of how to add matrices as though writing to someone who knows what a matrix is but not much more.
T/F: There is only 1 zero matrix.
T/F: To multiply a matrix by 2 means to multiply each entry in the matrix by 2.

2.2
T/F: Column vectors are used more in this text than row vectors, although some other texts do the opposite.
T/F: To multiply $\tta\times\ttb$, the number of rows of \tta\ and \ttb\ need to be the same.
T/F: The entry in the 2$^\text{nd}$ row and 3$^\text{rd}$ column of the product \tta\ttb\ comes from multipling the 2$^\text{nd}$ row of \tta\ with the 3$^\text{rd}$ column of \ttb.
Name two properties of matrix multiplication that also hold for ``regular multiplication'' of numbers.
Name a property of ``regular multiplication'' of numbers that does not hold for matrix multiplication.
T/F: $\tta^3 = \tta\cdot\tta\cdot\tta$

2.3
T/F: Two vectors with the same length and direction are equal even if they start from different places.
T/F: Adding two vectors together forms a parallelogram.
T/F: Multiplying a vector by 2 doubles its length.
What do mathematicians do?
T/F: Mutliplying a vector by a  matrix always changes its length and direction

2.4
T/F: The equation \ttaxb\ is just another way of writing a system of linear equations.
T/F: In solving \ttaxo, if there are 3 free variables, then the solution will be ``pulled apart'' into 3 components.
T/F: A homogeneous system of linear equations is one in which all of the coefficients are 0.
Whether or not the equation \ttaxb\ has a solution depends on an intrinsic property of \underline{\hskip .5in}.

2.5
T/F: To solve the matrix equation $\tta\ttx=\ttb$, put the matrix $\bmx{cc} \tta & \ttx \emx$ into \rref\ and interpret the result properly.
T/F: The first column of a matrix product \tta\ttb\ is \tta\ times the first column of \ttb.
Give two reasons why one might solve for the columns of \ttx\ in the equation \tta\ttx=\ttb\ separately.

2.6
T/F: If \tta\ and \ttb\ are square matrices where \tta\ttb=\tti, then \ttb\tta=\tti.
T/F: A matrix \tta\ has exactly one inverse, infinite inverses, or no inverse.
T/F: Everyone is special.
T/F: If \tta\ is invertible, then \ttaxo\ has exactly 1 solution.
What is a corollary?
Computing the inverse of a matrix is a very useful operation to know how to do.

2.7
What does it mean to say that two statements are ``equivalent?''
T/F: If \tta\ is not invertible, then \ttaxo\ could have no solutions.
T/F: If \tta\ is not invertible, then \ttaxb\ could have infinite solutions.
What is the inverse of the inverse of \tta?
T/F: Solving \ttaxb\ using Gaussian elimination is faster than using the inverse of \tta.

3.1
T/F: If \tta\ is a $3\times 5$ matrix, then \ttat\ will be a $5\times 3$ matrix.
T/F: An upper triangular matrix has only zeros above the diagonal.
T/F: A matrix is symmetric if it doesn't change when you take its transpose.
What is the transpose of the transpose of \tta?
Give 2 other terms to describe symmetric matrices besisdes ``interesting.''

3.2
T/F: We only compute the trace of square matrices.
T/F: One can tell if a matrix is invertible by computing the trace.

3.3
T/F: The determinant of a matrix is always positive.
T/F: To compute the determinant of a $3\times 3$ matrix, one needs to compute the determinants of 3 $2\times 2$ matrices.
T/F: The determinant of a matrix can be 0.

3.4
T/F: Having the choice to compute the determinant of a  matrix using cofactor expansion along any row or column is most useful when there are lots of zeros in a row or column.
Which elementary row operation does not change the determinant of a matrix?
Why do mathematicians rarely smile?
T/F: When computers are used to compute the determinant of a matrix, cofactor expansion is rarely used.

3.5
T/F: Cramer's Rule is another method to compute the determinant of a matrix.
T/F: Cramer's Rule is often used because it is more efficient than Gaussian elimination.
Mathematicians use what word to describe the connections between seemingly unrelated ideas?


4.1
T/F: Given any matrix \tta, we can always find a vector \vx\ where $\tta\vx=\vx$.
T/F: The zero vector is an \ev\ for every matrix \tta.
T/F: If \tta\ is a $5\times 5$ matrix, to find the \el s of \tta, we would need to find the roots of a 5$^\text{th}$ degree polynomial.

4.2
T/F: \tta\ and \tta\ have the same \ev s.
T/F: \tta\ and \ttai\ have the same \el s.
T/F: Marie Ennemond Camille Jordan was a guy.
T/F: Matrices with a trace of 0 are important, although we haven't seen why.
T/F: A matrix \tta\ is invertible only if 1 is an \el\ of \tta.

5.1
T/F: Transforming the Cartesian plane through matrix multiplication makes curved lines straight.
T/F: To understand how the Cartesian plane is affected by multiplication by a matrix, it helps to know how the unit square is affected.
T/F: If one draws a picture of a sheep on the Cartesian plane, then transformed the plane using the matrix $$\bmx{cc} -1&0\\0&1\emx,$$ one could say that the sheep was ``sheared.''

5.2
T/F: Translating the Cartesian plane 2 units up is a linear transformation.
T/F: If $T$ is a linear transformation, then $T(\zero)=\zero$.

5.3
T/F: The viewpoint of the reader makes a difference in how vectors in 3D look.
T/F: If two vectors are not near each other, then they will not appear to be near each other when graphed. 
T/F: The parallelogram law only applies to adding vectors in 2D.










